% coding:utf-8

%E-Wall
%Copyright (C) 2013, Daniel Winz, Ervin Mazlagic

%This program is free software; you can redistribute it and/or
%modify it under the terms of the GNU General Public License
%as published by the Free Software Foundation; either version 2
%of the License, or (at your option) any later version.

%This program is distributed in the hope that it will be useful,
%but WITHOUT ANY WARRANTY; without even the implied warranty of
%MERCHANTABILITY or FITNESS FOR A PARTICULAR PURPOSE.  See the
%GNU General Public License for more details.
%----------------------------------------

\section{Detailkonzept}

\subsection{Distanzmessung}
Die Distanzmessung erfolgt mittels eines Ultraschallsensors vom Typ HC-SR04. 
Da dieser Sensor eine Betriebsspannung von 5 V benötigt, wird er direkt von 
der vorhandenen 5V Leitung gespiesen. Die Kommunikation mit dem Mikrocontroller 
erfolgt über Pegelwandler. Die Distanzmessung erfolgt über die Laufzeit eines 
ausgestrahlten Ultraschallimpulses. Dieser wird durch ein High mit einer 
Länge von 10 $\mu$s auf dem Anschluss "'Trig"' ausgelöst. Anschliessend treibt 
der Sensor das Signal "'Echo"' High. Die Länge dieses Pulses entspricht dabei 
der Laufzeit des Ultraschallsignals. Die Distanz errechnet sich daraus wie 
folgt: 
\[ d = \frac{t_{echo} \cdot v_s}{2} \]
\begin{tabular}{@{}ll}
  $d$: & Distanz [m]\\
  $t_{echo}$: & Laufzeit [s]\\
  $v_s$: & Schallgeschwindigkeit [m/s]
\end{tabular}

\begin{figure}[h!]
  \center
  \begin{tikztimingtable}
    {Trigger} & LH10L\\
    {Ultraschall} & 2L16{0.25c}5L;[gray]16{0.25c};L\\
    {Echo} & 4L7HL\\
  \end{tikztimingtable}
\end{figure}

\subsection{Ansteuerung Lüfter / LED}
