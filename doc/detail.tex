% coding:utf-8

%E-Wall
%Copyright (C) 2013, Daniel Winz, Ervin Mazlagic

%This program is free software; you can redistribute it and/or
%modify it under the terms of the GNU General Public License
%as published by the Free Software Foundation; either version 2
%of the License, or (at your option) any later version.

%This program is distributed in the hope that it will be useful,
%but WITHOUT ANY WARRANTY; without even the implied warranty of
%MERCHANTABILITY or FITNESS FOR A PARTICULAR PURPOSE.  See the
%GNU General Public License for more details.
%----------------------------------------

\section{Detailkonzept}

\subsection{Speisung}
Als Speisung wird ein ATX Netzteil verwendet. Jedoch werden nur die Spannungen 
12V und 5V verwendet. Um den Leerlaufbetrieb des Netzteils und damit ein 
unkontrolliertes Hochlaufen der Spannungen zu vermeiden, wird die Leitung 5V 
mit einem Lastwiderstand belastet. Der Mikrocontroller wird mit der 
Standbyspannung des Netzteils versorgt um die Leistungsaufnahme im Standby zu 
minimieren. Die Versorgung erfolgt dabei über den USB Anschluss des 
Launchpads. Die stabilisierte Betriebsspannung für den Mikrocontroller erzeugt 
der im Launchpad integrierte Spannungsregler. Bei Bedarf wird das Netzteil vom 
Controller gestartet. 

\subsection{Distanzmessung}
Die Distanzmessung erfolgt mittels eines Ultraschallsensors vom Typ HC-SR04. 
Da dieser Sensor eine Betriebsspannung von 5 V benötigt, wird er direkt von 
der vorhandenen 5V Leitung gespiesen. Die Kommunikation mit dem Mikrocontroller 
erfolgt über Pegelwandler. Die Distanzmessung erfolgt über die Laufzeit eines 
ausgestrahlten Ultraschallimpulses. Dieser wird durch ein High mit einer 
Länge von 10 $\mu$s auf dem Anschluss "'Trig"' ausgelöst. Anschliessend treibt 
der Sensor das Signal "'Echo"' High. Die Länge dieses Pulses entspricht dabei 
der Laufzeit des Ultraschallsignals. Die Distanz errechnet sich daraus wie 
folgt: 
\[ d = \frac{t_{echo} \cdot v_s}{2} \]
\begin{tabular}{@{}ll}
  $d$: & Distanz [m]\\
  $t_{echo}$: & Laufzeit [s]\\
  $v_s$: & Schallgeschwindigkeit [m/s]
\end{tabular}

\begin{figure}[h!]
  \center
  \begin{tikztimingtable}
    Trigger     & L0.25C10C\\
    Ultraschall & 1.25L16{0.25c}5L;[gray]16{0.25c};L\\
    Echo        & 3.25L7CC\\
  \end{tikztimingtable}
  \label{tim_dist}
  \caption{Zeitdiagramm der Ansteuerung des Ultraschallsensors HC-SR04}
\end{figure}

\subsection{Ansteuerung Lüfter / LED}
Die Lüfter und LEDs werden zunächst nur geschaltet. Die Lüfter werden dabei 
durch das Schalten der Betriebsspannung gesteuert. Die LED werden in kleine 
Gruppen aufgeteilt, um die Gruppen einzeln schalten zu können. 
