% coding:utf-8

%E-Wall
%Copyright (C) 2013, Daniel Winz, Ervin Mazlagic

%This program is free software; you can redistribute it and/or
%modify it under the terms of the GNU General Public License
%as published by the Free Software Foundation; either version 2
%of the License, or (at your option) any later version.

%This program is distributed in the hope that it will be useful,
%but WITHOUT ANY WARRANTY; without even the implied warranty of
%MERCHANTABILITY or FITNESS FOR A PARTICULAR PURPOSE.  See the
%GNU General Public License for more details.
%----------------------------------------

\section{Anforderungen}
Es soll ein Gerät entwickelt werden, mit welchem die Anwesenheit von Personen 
erkannt werden soll. Wird eine Person erkannt, so sollen mehrere Lüfter und 
LEDs dies anzeigen. Dabei sollen folgende Anforderungen erfüllt werden. 

\begin{table}[h!]
  \begin{tabular}{@{}p{0.3\textwidth}p{0.2\textwidth}p{0.2\textwidth}}
    \rowcolor{white} \textbf{Anforderung}     & \textbf{Pflicht} & \textbf{Wunsch}\\
    \rowcolor{lgray} Speisung                 & ATX Netzteil     &                \\
    \rowcolor{lgray}                          &                  & abschaltbar    \\
    \rowcolor{white} Minimale Messdistanz     & 3 Meter          & 5 Meter        \\
    \rowcolor{lgray} Anzahl steuerbarer Lüfter& 12               &                \\
    \rowcolor{white} Drehzahl einstellbar     &                  & x              \\
    \rowcolor{lgray} Anzahl LED               & 12               & 24             \\
    \rowcolor{white} jede LED einzeln steuerbar &                & x              \\
    \rowcolor{lgray} erweiterbar              &                  & x              \\
  \end{tabular}
  %\caption{Anforderungen}
  \label{tab:anforderungen}
\end{table}
